% Options for packages loaded elsewhere
\PassOptionsToPackage{unicode}{hyperref}
\PassOptionsToPackage{hyphens}{url}
%
\documentclass[
]{book}
\title{Mappeeksamen}
\author{Margit Dahl Sørensen}
\date{2021-12-01}

\usepackage{amsmath,amssymb}
\usepackage{lmodern}
\usepackage{iftex}
\ifPDFTeX
  \usepackage[T1]{fontenc}
  \usepackage[utf8]{inputenc}
  \usepackage{textcomp} % provide euro and other symbols
\else % if luatex or xetex
  \usepackage{unicode-math}
  \defaultfontfeatures{Scale=MatchLowercase}
  \defaultfontfeatures[\rmfamily]{Ligatures=TeX,Scale=1}
\fi
% Use upquote if available, for straight quotes in verbatim environments
\IfFileExists{upquote.sty}{\usepackage{upquote}}{}
\IfFileExists{microtype.sty}{% use microtype if available
  \usepackage[]{microtype}
  \UseMicrotypeSet[protrusion]{basicmath} % disable protrusion for tt fonts
}{}
\makeatletter
\@ifundefined{KOMAClassName}{% if non-KOMA class
  \IfFileExists{parskip.sty}{%
    \usepackage{parskip}
  }{% else
    \setlength{\parindent}{0pt}
    \setlength{\parskip}{6pt plus 2pt minus 1pt}}
}{% if KOMA class
  \KOMAoptions{parskip=half}}
\makeatother
\usepackage{xcolor}
\IfFileExists{xurl.sty}{\usepackage{xurl}}{} % add URL line breaks if available
\IfFileExists{bookmark.sty}{\usepackage{bookmark}}{\usepackage{hyperref}}
\hypersetup{
  pdftitle={Mappeeksamen},
  pdfauthor={Margit Dahl Sørensen},
  hidelinks,
  pdfcreator={LaTeX via pandoc}}
\urlstyle{same} % disable monospaced font for URLs
\usepackage{longtable,booktabs,array}
\usepackage{calc} % for calculating minipage widths
% Correct order of tables after \paragraph or \subparagraph
\usepackage{etoolbox}
\makeatletter
\patchcmd\longtable{\par}{\if@noskipsec\mbox{}\fi\par}{}{}
\makeatother
% Allow footnotes in longtable head/foot
\IfFileExists{footnotehyper.sty}{\usepackage{footnotehyper}}{\usepackage{footnote}}
\makesavenoteenv{longtable}
\usepackage{graphicx}
\makeatletter
\def\maxwidth{\ifdim\Gin@nat@width>\linewidth\linewidth\else\Gin@nat@width\fi}
\def\maxheight{\ifdim\Gin@nat@height>\textheight\textheight\else\Gin@nat@height\fi}
\makeatother
% Scale images if necessary, so that they will not overflow the page
% margins by default, and it is still possible to overwrite the defaults
% using explicit options in \includegraphics[width, height, ...]{}
\setkeys{Gin}{width=\maxwidth,height=\maxheight,keepaspectratio}
% Set default figure placement to htbp
\makeatletter
\def\fps@figure{htbp}
\makeatother
\setlength{\emergencystretch}{3em} % prevent overfull lines
\providecommand{\tightlist}{%
  \setlength{\itemsep}{0pt}\setlength{\parskip}{0pt}}
\setcounter{secnumdepth}{5}
\usepackage{booktabs}
\ifLuaTeX
  \usepackage{selnolig}  % disable illegal ligatures
\fi
\usepackage[]{natbib}
\bibliographystyle{plainnat}

\begin{document}
\maketitle

{
\setcounter{tocdepth}{1}
\tableofcontents
}
\hypertarget{rabilitet}{%
\chapter{Rabilitet}\label{rabilitet}}

Placeholder

\hypertarget{introduksjon}{%
\section{Introduksjon}\label{introduksjon}}

\hypertarget{metode}{%
\section{Metode}\label{metode}}

\hypertarget{resultater}{%
\section{Resultater}\label{resultater}}

\hypertarget{diskusjon}{%
\section{Diskusjon}\label{diskusjon}}

\hypertarget{labrapport}{%
\chapter{Labrapport}\label{labrapport}}

\hypertarget{formuxe5l}{%
\section{Formål}\label{formuxe5l}}

\hypertarget{arbeidskrav-i-vitenskapsteori}{%
\chapter{Arbeidskrav i vitenskapsteori}\label{arbeidskrav-i-vitenskapsteori}}

Placeholder

\hypertarget{oppgave-1}{%
\section{Oppgave 1}\label{oppgave-1}}

\hypertarget{oppgave-2}{%
\section{Oppgave 2}\label{oppgave-2}}

\hypertarget{oppgave-3}{%
\section{Oppgave 3}\label{oppgave-3}}

\hypertarget{styrke-og-utholdenhetstrening-for-godt-trente-syklister}{%
\chapter{Styrke og utholdenhetstrening for godt trente syklister}\label{styrke-og-utholdenhetstrening-for-godt-trente-syklister}}

Styrketrening og utholdenhetstrening for godt trente syklister Introduksjon I denne oppgaven skal jeg se på fem ulike forskningsartikler som undersøke hvilken effekt styrketrening, i tillegg til utholdenhetstrening, har på godt trente syklister. Disse fem artiklene har jeg valgt på bakgrunn av at alle ser på hvilken effekt styrketrening har på prestasjon i sykkel, og hvordan de har brukt ulike metoder og tester for å vise resultatet.

\hypertarget{metode-1}{%
\section{Metode}\label{metode-1}}

\underline{\textbf{Hypotese}} Litteraturen som er valgt til denne oppgaven har alle som hovedmål å undersøke hvordan tung styrketrening i tillegg til vanlig utholdenhetstrening vil påvirke ulike prestasjonsfaktorer innenfor sykling. Hvilke prestasjonsfaktorer som blir vektlagt varierer mellom de ulike forskningsartiklene. Rønnestad \citep{rønnestad2010b} presenterer sin hypotese at styrketreningen vil påvirke muskeltverrsnitt i lårmuskulaturen, power output, windgate test, og 40 min all-out test. Vikmoen et al \citep{vikmoen2016} presenterer samme hypotese, men studien gjennomføres kun på kvinnelige elitesyklister. I artikkelen fra \citep{rønnestad2010a} har Rønnestad et al en hypotese om at vedlikehold av tung styrketrening gjennom starten av treningssesongen vil positivt påvirke utholdenheten over lengre konkurranser i slutten av en 13 ukers periode. Dette minne om hypotesen Aagaard 2011 legger frem, men intervensjonen og resultater blir gjennomført i forberedelses sesong, og ikke i konkurransesesong. Siste artikkel har en hypotese som sier at 10 uker med tung styrketrening sammen med utholdenhetstrening, og 15 uker med vedlikehold av styrketreningen vil gi økt styrke i pedaltråkk og øke styrken i benmuskulaturen \citep{rønnestad2015}.

\underline{\textbf{Studie design}}

Forskningsartiklene som er brukt i denne oppgaven har som hovedmål å finne ut hvordan styrketrening, sammen med utholdenhetstrening, påvirker profesjonelle eller godt trente syklisters prestasjon. Felles for alle studiene er at de gjennomføre en styrkeintervensjon fra 11 til 25 uker på intervensjonsgruppene, og sammenligner forskjellene i ulike styrke- og utholdenhetstester med en kontrollgruppe. De sammenligner Dette vil si at alle studiene bruker randomisert kontrollert studie {[}RCT{]} som studiedesign. I et RCT blir deltakerne tilfeldig valgt hvilken gruppe de tilhører, kontrollgruppe eller intervensjonsgruppe. Ved å tilfeldig fordele de ulike deltakerne i kontrollgruppe, og intervensjonsgruppe vil man minske sannsynligheten for at det er store forskjeller mellom gruppene \citep{hulley2013, Parab2010}. Det er med fordel at alle studiene i denne oppgaven bruker dette studie designet for å kunne se effekten for intervensjonsgruppen, mot kontrollgruppen. Dette vil si at resultatene vil vise hvilken effekt intervensjonen har eller ikke har, og ikke være påvirket av store gruppeforskjeller \citep{helsebiblioteketuå} . Utvalg: I de ulike studiene skilte deltaker gruppene seg fra 12 til 23 deltakere. Alle deltakerne var godt trente syklister. I noen av prosjektene var deltakerne syklister på høyt nasjonalt eller internasjonalt nivå \citep{rønnestad2010a, rønnestad2010b, rønnestad2015, aagaard2011}. Ingen av forfatterne beskrev rekrutteringsprosessen eller at det ble gjennomført en styrkeberegning (antall beregning) for å argumentere for antall deltakere i studiene.

\underline{\textbf{Tester}}

Alle forskningsprosjektene gjennomført av Rønnestad \citep{rønnestad2010a, rønnestad2010b, rønnestad2015} gjennomfører de samme fysiske testene for å teste intervensjonsgruppen mot kontrollgruppen. Disse testene er beskrevet i tabell 1. Vo2 maks test, windgate og 40 min all-out test ble brukt i prosjektet gjennomført av Vikmoen \citep{vikmoen2016}. I det siste prosjektet ble det gjennomført en 5 min prestasjonstest for å se på effekten av intervensjonen ved kort utholdenhetsprestasjon \citep{aagaard2011}. Det ble og gjennomført en 45 min utholdenhetstest. Felles for alle forskningsprosjektene er at testene var standardisert, for å øke validiteten. Deltakerne måtte følge visse regler under testdagene slik at de eksterne faktorene skulle være så like som mulige. Dette innebar regulasjoner på inntak av mat og drikke, og trening under testdagene. Testtidspunkt var standardisert til den enkelte deltaker, slik at testene ble gjennomført likt på dagen, og temperaturen i testrommet var standardisert på alle utøverne.

De ulike artiklene har brukt forskjellige statistiske tester, for å regne ut forskjellen mellom gruppene. For å regne ut forskjellen mellom gruppene ved pre-test har fire av prosjektene brukt uparret t-test \citep{rønnestad2010a, rønnestad2010b, rønnestad2015, vikmoen2016}. En t-test blir brukt for å regne ut om det er signifikant forskjell mellom gjennomsnittet til to grupper. I disse studiene er t-testen brukt på to uavhengige grupper, som gjør at man må bruke en uparet t-test \citet{kim2015} . I motsetning til de andre har Aagaard \citep{aagaard2011} gjennomført en ikke parametisk ANOVA analyse for å regne ut forskjellen mellom gruppene ved pre-test. Ikke parametiske analyser brukes for å sammenligne data som ikke møter standarden for nomralditrubisjon, eller for mindre grupper som Aagaard har (n=14) \citep{altman2009}. Enveis ANOVA er en variasjonsanalyse som regner ut forskjellen i gjennomsnitt mellom to ulike grupper grupper. Hvis dette gir en p-verdi som er lavere enn det gitte signifikans nivået kan man konkludere med at det er forskjell mellom gruppene, men man kan ikke si hvilken forskjell det er. For å finne denne må man gjennomføre en post hoc test for å finne hvilke grupper som er ulike (kilde). For å regne ut den relative forskjellen mellom intervensjonsgruppen og kontrollgruppen har Aagard også brukt en ikke-parametisk ANOVA med post hoc. Vikmoen \citep{vikmoen2016} og Rønnestad \citep{rønnestad2015} brukte uparet t-test for å regne ut relativ forskjell mellom gruppene. Rønnestad \citep{rønnestad2010a, rønnestad2010b} gjennomførte en toveis ANOVA med Bonferroni for å regne ut den relative forskjellen ved post-test. Ved å bruke bonferroni vil de slippe å gjennomføre flere t-tester og få flere p-verdier

Resultater \citep{rønnestad2010b} økte snittwatten med 6.0 +- 1.7\%, Men det var en tendens til høyere snittwatt i kontrollgruppen (4.6 ± 2.0\%). Det var ingen signifikant endring mellom gruppene, og det var kun kontrollgruppen som forbedret watten signifikant. Den maksimal styrken økte med 21.2 ± 4.9\% i kontrollgruppen. Det var en signifikant forskjell mellom gruppene i maksimal styrke. \citet{rønnestad2010a} Resultatene viser en signifikant forskjell mellom intervensjonsgruppen og kontrollgruppen etter endt intervensjon. Intervensjonsgruppen økte prestasjonen med 14 ±3\%, og kontrollgruppen økte 4 ±1\% \citet{rønnestad2015} Intervensjonsgruppen økte snittwatten med 6,5 ±5,7\%. Gruppen som bare trente utholdenhet, hadde ingen endring i prestasjon. Det var en signifikant økning i intervensjonsgruppen i forhold til kontrollgruppen \citet{aagaard2011} Etter endt studie økte intervensjonsgruppen prestasjonen med 8\% på 45 min prestasjonstesten. På 45 minutters prestasjonstest. Forfatterne konkluderer med at det var en signifikant forskjell mellom de to ulike gruppene på prestasjonstesten. \citet{vikmoen2016} Intervensjonsgruppen hadde en økning 6,4\% ± 7,9 \% på 40 minutters prestasjonstest. Det var ingen signifikant endring i kontrollgruppen 2,0 ±6,1\%). Ingen signifikant forskjell mellom gruppene.

\hypertarget{ett-sett-vs-tre-sett-styrketrening}{%
\chapter{Ett sett vs tre sett styrketrening}\label{ett-sett-vs-tre-sett-styrketrening}}

Placeholder

\hypertarget{introduksjon-1}{%
\section{Introduksjon}\label{introduksjon-1}}

\hypertarget{resultater-1}{%
\section{Resultater}\label{resultater-1}}

\hypertarget{diskusjon-1}{%
\section{Diskusjon}\label{diskusjon-1}}

\hypertarget{konklusjon}{%
\section{Konklusjon}\label{konklusjon}}

  \bibliography{book.bib,packages.bib}

\end{document}
