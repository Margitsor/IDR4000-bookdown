% Options for packages loaded elsewhere
\PassOptionsToPackage{unicode}{hyperref}
\PassOptionsToPackage{hyphens}{url}
%
\documentclass[
]{book}
\title{Mappeeksamen}
\author{Margit Dahl Sørensen}
\date{2021-12-01}

\usepackage{amsmath,amssymb}
\usepackage{lmodern}
\usepackage{iftex}
\ifPDFTeX
  \usepackage[T1]{fontenc}
  \usepackage[utf8]{inputenc}
  \usepackage{textcomp} % provide euro and other symbols
\else % if luatex or xetex
  \usepackage{unicode-math}
  \defaultfontfeatures{Scale=MatchLowercase}
  \defaultfontfeatures[\rmfamily]{Ligatures=TeX,Scale=1}
\fi
% Use upquote if available, for straight quotes in verbatim environments
\IfFileExists{upquote.sty}{\usepackage{upquote}}{}
\IfFileExists{microtype.sty}{% use microtype if available
  \usepackage[]{microtype}
  \UseMicrotypeSet[protrusion]{basicmath} % disable protrusion for tt fonts
}{}
\makeatletter
\@ifundefined{KOMAClassName}{% if non-KOMA class
  \IfFileExists{parskip.sty}{%
    \usepackage{parskip}
  }{% else
    \setlength{\parindent}{0pt}
    \setlength{\parskip}{6pt plus 2pt minus 1pt}}
}{% if KOMA class
  \KOMAoptions{parskip=half}}
\makeatother
\usepackage{xcolor}
\IfFileExists{xurl.sty}{\usepackage{xurl}}{} % add URL line breaks if available
\IfFileExists{bookmark.sty}{\usepackage{bookmark}}{\usepackage{hyperref}}
\hypersetup{
  pdftitle={Mappeeksamen},
  pdfauthor={Margit Dahl Sørensen},
  hidelinks,
  pdfcreator={LaTeX via pandoc}}
\urlstyle{same} % disable monospaced font for URLs
\usepackage{longtable,booktabs,array}
\usepackage{calc} % for calculating minipage widths
% Correct order of tables after \paragraph or \subparagraph
\usepackage{etoolbox}
\makeatletter
\patchcmd\longtable{\par}{\if@noskipsec\mbox{}\fi\par}{}{}
\makeatother
% Allow footnotes in longtable head/foot
\IfFileExists{footnotehyper.sty}{\usepackage{footnotehyper}}{\usepackage{footnote}}
\makesavenoteenv{longtable}
\usepackage{graphicx}
\makeatletter
\def\maxwidth{\ifdim\Gin@nat@width>\linewidth\linewidth\else\Gin@nat@width\fi}
\def\maxheight{\ifdim\Gin@nat@height>\textheight\textheight\else\Gin@nat@height\fi}
\makeatother
% Scale images if necessary, so that they will not overflow the page
% margins by default, and it is still possible to overwrite the defaults
% using explicit options in \includegraphics[width, height, ...]{}
\setkeys{Gin}{width=\maxwidth,height=\maxheight,keepaspectratio}
% Set default figure placement to htbp
\makeatletter
\def\fps@figure{htbp}
\makeatother
\setlength{\emergencystretch}{3em} % prevent overfull lines
\providecommand{\tightlist}{%
  \setlength{\itemsep}{0pt}\setlength{\parskip}{0pt}}
\setcounter{secnumdepth}{5}
\usepackage{booktabs}
\usepackage{booktabs}
\usepackage{longtable}
\usepackage{array}
\usepackage{multirow}
\usepackage{wrapfig}
\usepackage{float}
\usepackage{colortbl}
\usepackage{pdflscape}
\usepackage{tabu}
\usepackage{threeparttable}
\usepackage{threeparttablex}
\usepackage[normalem]{ulem}
\usepackage{makecell}
\usepackage{xcolor}
\usepackage{fontspec}
\usepackage{multicol}
\usepackage{hhline}
\usepackage{hyperref}
\ifLuaTeX
  \usepackage{selnolig}  % disable illegal ligatures
\fi
\usepackage[]{natbib}
\bibliographystyle{plainnat}

\begin{document}
\maketitle

{
\setcounter{tocdepth}{1}
\tableofcontents
}
\hypertarget{rabilitet}{%
\chapter{Rabilitet}\label{rabilitet}}

\hypertarget{introduksjon}{%
\section{Introduksjon}\label{introduksjon}}

Maksimalt oksygenopptak \(\dot VO_{2max}\) ble først beskrevet av
{[}@ \citep{hill1923} , og kan defineres som kroppens evne til å ta opp og
forbruke oksygen per tidsenhet \citep{HillLupton, BassetHowley}. Innen
toppidrett måles ofte det maksimale oksygenopptaket for å måle utøverens
kapasitet opp mot arbeidskravet i den spesifikke idretten, og
\(\dot VO_{2max}\) kan i så måte også sees på som et mål på den aerobe
effekten til utøveren \citep{bassett2000}. I Olympiatoppens testprotokoller
benytter de flere definerte hjelpekriterier for å sikre at man faktisk
har funnet deltakerens maksimale oksygenopptak \citep{tønnessen2017}.
Følgende kriterier er beskrevet; platå i \(\dot VO_{2max}\) er oppnådd,
økning i ventilasjon med utflating av \(\dot VO_{2max}\) verdi, RER over
1.10/1.05, og blodlaktat over 8 \citep{tønnessen2017}
\citep{røn}
\#\# Metode

I forkant av testen målte alle deltakerne kroppsvekten i samme klær som
ble brukt under testen, men ble bedt om å ta av seg skoene. Kroppsvekten
som senere brukt i beregningen av maksimalt oksygenopptak (ml kg\textsuperscript{-1}
min\textsuperscript{-1}) er kroppsvekten målt i forkant av test, etter at 300g har blitt
trukket av for å ta høyde for vekten av klærne. For å sikre intern
validitet ble deltakerne bedt om å avstå fra anstrengende fysisk
aktivitet dagen før test, standardisere måltidet i forkant av test samt
avstå fra inntak av koffein under de siste 12 timene før testen
\citep{halperin2015}. Pre- og post-tester ble gjennomført på samme tid på
døgnet under standardiserte forhold.. Post-test ble gjennomført 6 dager
etter gjennomført
pretest.
Det ble ikke kontrollert for fysisk aktivitet mellom testdagene.

Alle deltakerne gjennomførte en 10 minutter lang oppvarmingsprotokoll på
tredemøllen (Woodway, 4 front , Wisconsin), beskrevet for deltakerne i
forkant av testen. Denne oppvarmingsprotokollen bestod av fem minutter
på 11-13 i Borg 6-20 RPE skala (Borg,1982),
etterfulgt av 2x1min på starthastighet og stigning med 30 sekund
mellom.
Siste tre min var også 11-13 i
borg. Etter oppvarming
var det to min pause før testen begynte. Starthastighet for begge
kjønn var satt til
8km/t, med stigning på 10.5\% og 5.5\% for henholdsvis menn og kvinner.

̇VO2max ble målt ved hjelp av en metabolsk analysator med
Vyntus CPX miksekammer (Vyntus CPX, Jaeger-CareFusion, UK). Forut for
alle tester ble analysatoren gass og volumkalibrert med en feillmargin
på henholdsvis 2\% og 0.2\%. Analysatoren ble
stilt inn til å gjøre målinger hvert 30sek, og V̇O\textsubscript{2max} ble kalkulert
gjennom å bruke snittet av de to høyeste påfølgende målingene av V̇O\textsubscript{2}.
Underveis i testen mottok alle deltakerne en høylytt verbal oppmuntring
fra testleder. Alle deltakerne
gjennomførte også begge testene med samme testleder og med samme
personer til stede i rommet for å redusere konfundering \citep{halperin}.

For hvert medgåtte minutt av testen ble hastigheten på møllen økt med
1km/t, helt til utmattelse, hvor testen ble avsluttet. Deltakernes
hjertefrekvens (GARMIN/POLAR) ble også registrert under hele testen. Når
testen ble avsluttet ble deltakerne bedt om å rapportere opplevd
anstrengelse ved hjelp av Borg-skala \citep{borg1982}. Maksimal
hjertefrekvens under testen ble også registrert. Ett minutt etter
avsluttet test ble hjertefrekvens registrert, og det ble målt og
analysert blodlaktat(BIOSEN).

\hypertarget{resultater}{%
\section{Resultater}\label{resultater}}

\providecommand{\docline}[3]{\noalign{\global\setlength{\arrayrulewidth}{#1}}\arrayrulecolor[HTML]{#2}\cline{#3}}

\setlength{\tabcolsep}{2pt}

\renewcommand*{\arraystretch}{1.5}

\begin{longtable}[c]{|p{1.08in}|p{1.02in}|p{1.02in}}



\hhline{>{\arrayrulecolor[HTML]{666666}\global\arrayrulewidth=2pt}->{\arrayrulecolor[HTML]{666666}\global\arrayrulewidth=2pt}->{\arrayrulecolor[HTML]{666666}\global\arrayrulewidth=2pt}-}

\multicolumn{1}{!{\color[HTML]{000000}\vrule width 0pt}>{\raggedright}p{\dimexpr 1.08in+0\tabcolsep+0\arrayrulewidth}}{\fontsize{11}{11}\selectfont{\textcolor[HTML]{000000}{\global\setmainfont{Helvetica}{}}}} & \multicolumn{1}{!{\color[HTML]{000000}\vrule width 0pt}>{\raggedright}p{\dimexpr 1.02in+0\tabcolsep+0\arrayrulewidth}}{\fontsize{11}{11}\selectfont{\textcolor[HTML]{000000}{\global\setmainfont{Helvetica}{Kvinner}}}} & \multicolumn{1}{!{\color[HTML]{000000}\vrule width 0pt}>{\raggedright}p{\dimexpr 1.02in+0\tabcolsep+0\arrayrulewidth}!{\color[HTML]{000000}\vrule width 0pt}}{\fontsize{11}{11}\selectfont{\textcolor[HTML]{000000}{\global\setmainfont{Helvetica}{Menn}}}} \\

\noalign{\global\setlength{\arrayrulewidth}{2pt}}\arrayrulecolor[HTML]{666666}\cline{1-3}

\endfirsthead

\hhline{>{\arrayrulecolor[HTML]{666666}\global\arrayrulewidth=2pt}->{\arrayrulecolor[HTML]{666666}\global\arrayrulewidth=2pt}->{\arrayrulecolor[HTML]{666666}\global\arrayrulewidth=2pt}-}

\multicolumn{1}{!{\color[HTML]{000000}\vrule width 0pt}>{\raggedright}p{\dimexpr 1.08in+0\tabcolsep+0\arrayrulewidth}}{\fontsize{11}{11}\selectfont{\textcolor[HTML]{000000}{\global\setmainfont{Helvetica}{}}}} & \multicolumn{1}{!{\color[HTML]{000000}\vrule width 0pt}>{\raggedright}p{\dimexpr 1.02in+0\tabcolsep+0\arrayrulewidth}}{\fontsize{11}{11}\selectfont{\textcolor[HTML]{000000}{\global\setmainfont{Helvetica}{Kvinner}}}} & \multicolumn{1}{!{\color[HTML]{000000}\vrule width 0pt}>{\raggedright}p{\dimexpr 1.02in+0\tabcolsep+0\arrayrulewidth}!{\color[HTML]{000000}\vrule width 0pt}}{\fontsize{11}{11}\selectfont{\textcolor[HTML]{000000}{\global\setmainfont{Helvetica}{Menn}}}} \\

\noalign{\global\setlength{\arrayrulewidth}{2pt}}\arrayrulecolor[HTML]{666666}\cline{1-3}\endhead



\multicolumn{3}{!{\color[HTML]{FFFFFF}\vrule width 0pt}>{\raggedright}p{\dimexpr 3.13in+4\tabcolsep+2\arrayrulewidth}!{\color[HTML]{FFFFFF}\vrule width 0pt}}{\fontsize{11}{11}\selectfont{\textcolor[HTML]{000000}{\global\setmainfont{Helvetica}{Verdier\ er\ gitt\ som\ gjennomsnitt\ og\ (Standardavvik)}}}} \\

\endfoot



\multicolumn{1}{!{\color[HTML]{000000}\vrule width 0pt}>{\raggedright}p{\dimexpr 1.08in+0\tabcolsep+0\arrayrulewidth}}{\fontsize{11}{11}\selectfont{\textcolor[HTML]{000000}{\global\setmainfont{Helvetica}{N}}}} & \multicolumn{1}{!{\color[HTML]{000000}\vrule width 0pt}>{\raggedright}p{\dimexpr 1.02in+0\tabcolsep+0\arrayrulewidth}}{\fontsize{11}{11}\selectfont{\textcolor[HTML]{000000}{\global\setmainfont{Helvetica}{4}}}} & \multicolumn{1}{!{\color[HTML]{000000}\vrule width 0pt}>{\raggedright}p{\dimexpr 1.02in+0\tabcolsep+0\arrayrulewidth}!{\color[HTML]{000000}\vrule width 0pt}}{\fontsize{11}{11}\selectfont{\textcolor[HTML]{000000}{\global\setmainfont{Helvetica}{7}}}} \\





\multicolumn{1}{!{\color[HTML]{000000}\vrule width 0pt}>{\raggedright}p{\dimexpr 1.08in+0\tabcolsep+0\arrayrulewidth}}{\fontsize{11}{11}\selectfont{\textcolor[HTML]{000000}{\global\setmainfont{Helvetica}{Alder\ (år)}}}} & \multicolumn{1}{!{\color[HTML]{000000}\vrule width 0pt}>{\raggedright}p{\dimexpr 1.02in+0\tabcolsep+0\arrayrulewidth}}{\fontsize{11}{11}\selectfont{\textcolor[HTML]{000000}{\global\setmainfont{Helvetica}{24.5\ (1.29)}}}} & \multicolumn{1}{!{\color[HTML]{000000}\vrule width 0pt}>{\raggedright}p{\dimexpr 1.02in+0\tabcolsep+0\arrayrulewidth}!{\color[HTML]{000000}\vrule width 0pt}}{\fontsize{11}{11}\selectfont{\textcolor[HTML]{000000}{\global\setmainfont{Helvetica}{23.9\ (1.77)}}}} \\





\multicolumn{1}{!{\color[HTML]{000000}\vrule width 0pt}>{\raggedright}p{\dimexpr 1.08in+0\tabcolsep+0\arrayrulewidth}}{\fontsize{11}{11}\selectfont{\textcolor[HTML]{000000}{\global\setmainfont{Helvetica}{Vekt\ (kg)}}}} & \multicolumn{1}{!{\color[HTML]{000000}\vrule width 0pt}>{\raggedright}p{\dimexpr 1.02in+0\tabcolsep+0\arrayrulewidth}}{\fontsize{11}{11}\selectfont{\textcolor[HTML]{000000}{\global\setmainfont{Helvetica}{58.9\ (6.28)}}}} & \multicolumn{1}{!{\color[HTML]{000000}\vrule width 0pt}>{\raggedright}p{\dimexpr 1.02in+0\tabcolsep+0\arrayrulewidth}!{\color[HTML]{000000}\vrule width 0pt}}{\fontsize{11}{11}\selectfont{\textcolor[HTML]{000000}{\global\setmainfont{Helvetica}{74.8\ (5.55)}}}} \\





\multicolumn{1}{!{\color[HTML]{000000}\vrule width 0pt}>{\raggedright}p{\dimexpr 1.08in+0\tabcolsep+0\arrayrulewidth}}{\fontsize{11}{11}\selectfont{\textcolor[HTML]{000000}{\global\setmainfont{Helvetica}{Høyde\ (cm)}}}} & \multicolumn{1}{!{\color[HTML]{000000}\vrule width 0pt}>{\raggedright}p{\dimexpr 1.02in+0\tabcolsep+0\arrayrulewidth}}{\fontsize{11}{11}\selectfont{\textcolor[HTML]{000000}{\global\setmainfont{Helvetica}{166\ (2.99)}}}} & \multicolumn{1}{!{\color[HTML]{000000}\vrule width 0pt}>{\raggedright}p{\dimexpr 1.02in+0\tabcolsep+0\arrayrulewidth}!{\color[HTML]{000000}\vrule width 0pt}}{\fontsize{11}{11}\selectfont{\textcolor[HTML]{000000}{\global\setmainfont{Helvetica}{180\ (3.1)}}}} \\

\noalign{\global\setlength{\arrayrulewidth}{2pt}}\arrayrulecolor[HTML]{666666}\cline{1-3}



\end{longtable}

Det var 11 deltakere i studien, samtlige deltakere er studenter ved
Høgskolen i Innlandet. Deskriptive data for disse deltakerne er vist i
Tabell 1, i Figur 1 kan man se utviklingen fra pre-test til post-test
fordelt på kjønn. Det typiske målefeilet (typical error, \citep{hopkins}) fra
pre til post-test er utregnet til å være 4.04\%.

\begin{figure}
\centering
\includegraphics{_main_files/figure-latex/unnamed-chunk-3-1.pdf}
\caption{\label{fig:unnamed-chunk-3}Figurtekst legg til\ldots{}}
\end{figure}

\hypertarget{diskusjon}{%
\section{Diskusjon}\label{diskusjon}}

Ettersom testing av maksimalt oksygenopptak er en test som gjennomføres
til utmattelse, vil man kunne forvente en viss variasjon i
testresultatene ettersom opplevd anstrengelse kan påvirkes av flere
ulike variabler \citep{halperin2015}. For å redusere
konfundering
vil flere faktorer være nyttig å ta hensyn til under slik testing. Som
nevnt i metoden vil standardisering av matinntak, koffeininntak, utstyr
og tidspunkt for gjennomføring av test være med på å kunne sikre intern
validitet i resultatene. Eksempler er deltakernes kjennskap til testen,
verbal oppmuntring og personer tilstede under testen er andre faktorer
som potensielt kan bidra til konfundering. Felles for alle faktorer er
at graden av påvirkning på resultatene muligens reduseres ved hjelp av
en standardisert testprotokoll. Deltakerne - og testlederne, sin
kjennskap til testen er en annen faktor som trolig påvirker resultatene
i vårt prosjekt. I dette tilfellet fantes det enkelte deltakere som
hadde gjennomført en liknende test flere ganger, og en kan da forvente
en mindre grad av variasjon mellom resultatene på pre og post test,
sammenlignet med de deltakerne som gjennomførte testen for første gang
på pretest. Dette fordi kjennskapen og kunnskapen de tilegnet seg på
pre-test, trolig spiller inn på testresultatene.
Standardfeilen
på 4.04\% kan også tyde på at enkelte av disse resultatene kan være
utsatt for konfundering av ulik sort \citep{hopkins2000}.

Grunnen til at vi snakker om standardfeil er at når vi ønsker å måle
påvirkningen av trening på en gruppe individer er det viktig å kunne si
noe om hva som er endring og hva som er støy (målefeil). Desto mindre
støy en test innebærer jo bedre er målingen. Målet som brukes er
standardfeil. Hva som danner denne variasjonen som representeres ved
typical error er multifaktorelt, men hoveddelen er som oftest biologisk
\citep{hopkins2000}.

For å måle standardfeil har vi brukt within subject deviation metoden.
Denne metoden påvirkes ikke av at gjennomsnittet endrer seg fra test til
test \citetext{\citealp[. Data for målinger i VO2max fra fem sertifiserte
Australske laboratorier fastslo ett gjennomsnitt på 2.2\% for
standardfeil \[@halperin2015\]. Data fra det Australske institutt for
sport har også fastslått at en standardfeil på omtrent 2\% er riktig for
både maksimal og submaksimal VO2 {[}\citet{clark2007}]{hopkins2000}; \citealp{robertson2010}; \citealp{saunders2009}}. Dette indikerer at med godt kalibrert utstyr og med
utøvere som er godt vant med testingen vil en standardfeil på 2\% for det
biologiske, og analytiske være riktig \citep{halperin2015}. Vår standardfeil
på 4.04\% kan derfor tenkes å være et bilde hvordan det kan se ut med få
deltakere, med ulikt utgangspunkt, men også uten skikkelig
standardisering av treningshverdagen i forkant av testene. Det kan også
tenkes at med et varierende nivå hos deltagerne kan enkelte oppleve en
treningseffekt av test 1. Samtidig som andre kanskje ble slitne av å få
en test inn i treningshverdagen.

\hypertarget{referanser}{%
\section{Referanser}\label{referanser}}

\hypertarget{labrapport}{%
\chapter{Labrapport}\label{labrapport}}

\hypertarget{formuxe5l}{%
\section{Formål}\label{formuxe5l}}

  \bibliography{book.bib,packages.bib}

\end{document}
